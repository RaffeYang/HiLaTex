\documentclass[a4paper,12pt]{article} % 文档类为 article,纸张为 A4,字号为 12pt
\usepackage{ctex} % 加载 ctex 宏包以支持中文
\usepackage{graphicx} % 加载 graphicx 包以支持插入图片
\usepackage{amsmath} % 加载 amsmath 包以支持数学公式
\usepackage{amsfonts} % 加载 amsfonts 包以支持数学字体
\usepackage{amssymb} % 加载 amssymb 包以支持数学符号
\usepackage{hyperref} % 加载 hyperref 包以支持超链接
\usepackage{geometry} % 加载 geometry 包以控制页面布局
\usepackage{algorithm} % 加载 algorithm 包以支持算法环境
\usepackage{algorithmic} % 加载 algorithmic 包以支持算法步骤
\usepackage{bookmark} % 加载 bookmark 宏包以支持书签
\usepackage{titlesec} % 加载 titlesec 包以自定义章节标题格式

% 页面设置
\geometry{top=2.5cm, bottom=2.5cm, left=2.5cm, right=2.5cm} % 设置页面边距

% 定义字体
\setCJKfamilyfont{wenkai}{LXGW WenKai} % 霞鹜文楷
\setCJKfamilyfont{heiti}{Noto Sans CJK SC} % 思源黑体
\setCJKfamilyfont{songti}{Noto Serif CJK SC} % 思源宋体
\setCJKfamilyfont{pingfang}{PingFang SC} % 苹方

% 定义新的命令来使用这些字体
\newcommand{\wenkaifont}{\CJKfamily{wenkai}} % 霞鹜文楷
\newcommand{\heitifont}{\CJKfamily{heiti}} % 思源黑体
\newcommand{\songtifont}{\CJKfamily{songti}} % 思源宋体
\newcommand{\pingfangfont}{\CJKfamily{pingfang}} % 苹方

% 自定义章节标题字体
\titleformat{\section}{\wenkaifont\huge\bfseries}{\thesection}{1em}{} % 设置 section 字体为霞鹜文楷
\titleformat{\subsection}{\heitifont\Large\bfseries}{\thesubsection}{1em}{} % 设置 subsection 字体为思源黑体
\titleformat{\subsubsection}{\songtifont\normalsize\bfseries}{\thesubsubsection}{1em}{} % 设置 subsubsection 字体为思源宋体

\title{\pingfangfont \LaTeX 中文模版} % 文档标题使用苹方
\author{\pingfangfont Raffe Yang} % 作者姓名使用苹方
\date{\today} % 日期

\begin{document}

\maketitle % 生成标题

\begin{abstract} % 摘要
    \pingfangfont 这是文档的摘要部分。摘要简要介绍了文档的内容和目的。
\end{abstract}

\tableofcontents % 生成目录

\section{引言} % 引言部分
\pingfangfont 引言部分介绍了文档的背景和研究的目的。可以在这里写一些基础知识。

\section{方法} % 方法部分
\pingfangfont 在这一部分,您可以详细描述您的研究方法。
\[
H(X) = -\sum_{i=1}^{n} P(x_i) \log_b P(x_i)
\]

\subsection{示例算法} % 算法部分
\begin{algorithm}
    \caption{示例算法}
    \begin{algorithmic}
        \STATE 输入:$x$
        \STATE 输出:$y$
        \STATE $y \gets x^2$
        \RETURN $y$
    \end{algorithmic}
\end{algorithm}

\section{结果} % 结果部分
\pingfangfont 在这一部分,您可以展示您的研究结果。

\subsection{结果分析}
\pingfangfont 这里可以添加对结果的分析。

\section{测试字体}
这是没有指定字体的效果
\wenkaifont 这是霞鹜文楷
\heitifont 这是思源黑体
\songtifont 这是思源宋体
\pingfangfont 这是苹方

\end{document}
