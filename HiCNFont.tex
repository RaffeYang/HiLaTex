\documentclass{article}
\usepackage{ctex}     % 加载 ctex 宏包
\author{Raffe Yang}
\title{\LaTeX 中文自定义字体显示}

% 定义字体
\setCJKfamilyfont{wenkai}{LXGW WenKai} % 霞鹜文楷
\setCJKfamilyfont{heiti}{Noto Sans CJK SC} % 思源黑体
\setCJKfamilyfont{songti}{Noto Serif CJK SC} % 思源宋体
\setCJKfamilyfont{pingfang}{PingFang SC} % 苹方

% 定义新的命令来使用这些字体
\newcommand{\wenkaifont}{\CJKfamily{wenkai}} % 霞鹜文楷
\newcommand{\heitifont}{\CJKfamily{heiti}} % 思源黑体
\newcommand{\songtifont}{\CJKfamily{songti}} % 思源宋体
\newcommand{\pingfangfont}{\CJKfamily{pingfang}} % 苹方

% 不建议设置mainfont,特别是CJKmainfont,因为这会提示警告。
% \setmainfont{PingFang SC} % 设置主英文字体为苹方

% \setCJKmainfont{PingFang SC} % 设置中文主字体为苹方
\begin{document}
\maketitle
\section{使用 macOS 字体的示例}
这是默认字体

\section{Example of using macOS fonts}
This is the default font.

\section{\LaTeX 内置中文显示}
中文输入。{\kaishu 这里是楷体显示},{\songti 这里是宋体显示},{\heiti 这里是黑体显示},{\fangsong 这里是仿宋显示}。

\section{霞鹜文楷}
{\wenkaifont 这是使用霞鹜文楷字体的中文文本。}

\section{思源黑体}
{\heitifont 这是使用思源黑体字体的中文文本。}

\section{思源宋体}
{\songtifont 这是使用思源宋体字体的中文文本。}

\section{苹方}
{\pingfangfont 这是使用苹方字体的中文文本。}

\section{欧拉公式}
欧拉Euler公式是数学中的一条重要公式。该公式表达了以自然对数为底的指数函数和三角函数之间的关系。欧拉公式如下:
\[
H(X) = -\sum_{i=1}^{n} P(x_i) \log_b P(x_i)
\]

\end{document}
